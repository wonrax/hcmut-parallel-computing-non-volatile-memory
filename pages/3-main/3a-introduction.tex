\section{Giới thiệu về đề tài}
\label{sec:introduction}

Trong nhiều năm nay, các công nghệ bộ nhớ (memory) đã thành công trong việc thu
nhỏ kích thước cell để đạt được mật độ cao hơn, tốc độ nhanh hơn với chi phí
thấp. Tuy nhiên, gần đây, sự phát triển của các công nghệ này đang dần tiến tới
giới hạn vật lý. Điều này đề khởi tầm quan trọng của việc phát triển các công
nghệ mới~\cite{apalkovSpintransferTorqueMagnetic2013}.

% Traditionally, disk drives or persistent memory devices were 
% employed to overcome DRAM limits. The modern attractive 
% option is to use solid-state drives (SSDs). However, even fastest 
% SSDs have orders-of-magnitude less bandwidth and higher 
% latency than DRAM; also, these devices usually operate data is 
% large blocks instead of bytes. (mironov)

Ổ cứng và NAND Flash được sử dụng để khắc phục giới hạn của DRAM. Giải pháp tối
ưu hiện nay là sử dụng solid-state drive (SSD), tuy nhiên, thiết bị SSD nhanh
nhất vẫn có băng thông thấp hơn DRAM gấp hàng nghìn
lần~\cite{mironovPerformanceEvaluationIntel2019}

Báo cáo này nghiên cứu về các công nghệ non-volatile memory, là các công nghệ
đang được phát triển và có nhiều tiềm năng thay thế các thiết bị hiện tại trong
tương lai. Trên cơ sở đó, báo cáo nghiên cứu xu hướng và sự phát triển của NVM
trong những năm gần đây. Cuối cùng là các ứng dụng và phân tích hiệu năng của
NVM trong tính toán hiệu năng cao.