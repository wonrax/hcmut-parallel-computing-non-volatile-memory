\section{Non-volatile memory}
\label{chap:non-volatile-memory}

Bộ nhớ (memory) là thiết bị dùng để lưu trữ dữ liệu cho máy tính. Bộ nhớ chính
(primary memory) được dùng cho các tác vụ yêu cầu tốc độ cao, bộ nhớ thứ cấp
(secondary memory) tuy có tốc độ chậm hơn nhưng có thể chứa được nhiều dữ liệu
hơn. Nếu cần thiết, các dữ liệu ở bộ nhớ chính có thể được lưu trữ ở bộ nhớ thứ
cấp thông qua kỹ thuật quản lý bộ nhớ gọi là bộ nhớ ảo (virtual
memory)~\cite{VolatileNonVolatileComputer}.

Trong các hệ thống tính toán ngày nay, bộ nhớ được chia thành hai loại:

\begin{itemize}
    \item \textbf{Volatile memory (bộ nhớ khả biến)}: là loại bộ nhớ cần điện
    năng để duy trì trạng thái các dữ liệu trong bộ nhớ. Đa số các bộ nhớ khả
    biến từ chất bán dẫn rơi vào hai loại: Static RAM (SRAM) hoặc dynamic RAM
    (DRAM). Đối với SRAM, ta chỉ cần duy trì nguồn điện thì dữ liệu sẽ được lưu
    giữ trong bộ nhớ, tuy nhiên SRAM cần sử dụng sáu bán dẫn (transistor) cho
    mỗi bit. Mặt khác, DRAM chỉ cần một bán dẫn cho mỗi bit, qua đó cho phép bộ
    nhớ có mật độ bit cao hơn. Tuy nhiên, DRAM cần các chu trình làm tươi
    (refresh cycle) thường xuyên để lưu giữ dữ liệu.
    
    \item \textbf{Non-volatile memory (bộ nhớ bất khả biến)}: là loại bộ nhớ có
    thể lưu trữ dữ liệu ngay cả khi thiết bị không được cung cấp điện năng.
    Điển hình cho bộ nhớ bất khả biến là các bộ nhớ từ (magnetic storage
    device, ví dụ như ổ cứng hay đĩa mềm), đĩa quang (optical disc) hay NAND
    Flash (điển hình là solid state drive hay SSD). Các bộ nhớ bất khả biến có
    tiềm năng trong tương lai bao gồm FeRAM, CBRAM,PRAM, SONOS, RRAM, Racetrack
    memory, NRAM và Millipede.
\end{itemize}

So với non-volatile memory, volatile memory là công nghệ có tốc độ vượt trội
hơn, tuy nhiên có khả năng mở rộng (scalability) kém hơn. Chính vì vậy, mục
đích của non-volatile memory là lưu trữ các dữ liệu mà CPU cần truy cập tức
thời. Trong khi đó, non-volatile memory được dùng để lưu trữ các dữ liệu sẽ
được sử dụng trong tương lai.

Khi chuyển dữ liệu từ NAND sang DRAM, hiệu suất chung của hệ thống bị giới hạn
bởi một độ trễ khá lớn do sự khác biệt giữa hai loại bộ nhớ. Sự khác biệt này
được giảm thiểu bằng các kỹ thuật sử dụng các kiến trúc khác nhau, qua đó phải
đánh đổi cho sự phức tạp của hệ thống và kích thước chip. Trong những năm gần
đây, các nhà nghiên cứu đã bắt đầu khám phá các cách thức cải tiến bộ nhớ cho
phù hợp với kiến trúc bộ nhớ hiện tại. Storage class memory (SCM, tạm dịch ``bộ
nhớ cấp lưu trữ'') được đề xuất để giảm sự khác biệt giữa ``memory-memory'' và
``storage-memory''~\cite{gouxOxRAMTechnologyDevelopment2019}.

Để đáp ứng các nhu cầu tính toán hiện nay, các thiết bị bộ nhớ mới được nghiên
cứu và ra đời. Nổi bật là spin-transfer torque magnetic RAM (STT-MRAM),
phase-change RAM (PCM), resistive RAM (ReRAM hay RRAM). Báo cáo này sẽ tập
trung nghiên cứu các công nghệ trên, cùng với NAND Flash là công nghệ đang
thống trị trên thị trường hiện tại.

\subsection{NAND Flash}
Bộ nhớ Flash được chế tạo bởi TS. Fujio Masuoka tại Toshiba Corp. vào năm 1984.
Dựa trên thiết kế của Masuoka, Intel đưa ra thị trường bộ nhớ NOR Flash để sử
dụng với mục đích lưu trữ code chương trình cho các sản phẩm hạng tiêu dùng.
NOR Flash cũng là nền tảng cho các thẻ nhớ Flash và các ổ đĩa bán dẫn (solid
state drive, hay SSD) vào những năm 1990.

Toshiba Corp. giới thiệu bộ nhớ NAND Flash vào năm 1988 với hứa hẹn đem lại giá
thành rẻ hơn NOR Flash và có thông lượng (throughput) nhanh hơn. Thay vì tổ
chức dưới dạng word hoặc byte như NOR Flash, bộ nhớ NAND Flash được tổ chức
dưới dạng trang (page) và xóa block (một block bao gồm 64 trang hoặc nhiều
hơn). Kiến trúc này có lợi cho giá thành nhưng không phù hợp với nhu cầu truy
cập ngẫu nhiên (random access). Vì vậy, NAND Flash thường được dùng để làm
thiết bị lưu trữ dữ liệu tương tự như đĩa quang hay ổ cứng.

\image[1]{img/nand_applications.png}{Các ứng dụng của NAND Flash (Nguồn: Forward
Insight).}{fig:nand_applications}

Nhu cầu về nhiếp ảnh và lưu trữ số là nguyên nhân chính làm tăng nhu cầu NAND
Flash trên thị trường tại thời điểm nó ra mắt. Ngoài ra, NAND Flash còn được
dùng để lưu trữ dữ liệu hay code chương trình ở các thiết bị bỏ túi như điện
thoại, máy nghe nhạc MP3, máy quay cầm tay, v.v.

Các thiết bị SSD đầu tiên là các thiết bị dựa trên RAM (RAM-based SSD) được chế
tạo tiên phong bởi StorageTek vào năm 1978. Đến những năm 1990s, SSD dựa trên
Flash (Flash-based SSD) mới được phát triển. Nhờ sự sụt giảm giá thành nhanh và
đáng kinh ngạc, NAND Flash đã có thể thâm nhập vào thị trường ổ cứng (HDD) và
làm tuyệt chủng công nghệ đĩa mềm (floppy disk). SSD có khả năng cung cấp tốc
độ nhanh hơn nhiều lần và có độ tin cậy cao nhờ thiết kế không phải chứa nhiều
các bộ phận vật lý.

Trong các môi trường tính toán hiệu năng cao, SSD có thể được sử dụng làm I/O
accelerator. Trong môi trường doanh nghiệp, các hệ thống HDD phải sử dụng các
kỹ thuật khác nhau để làm tăng băng thông và giảm thời gian truy cập dữ liệu.
Với hệ thống HDD sử dụng RAID, ta có thể làm tăng băng thông bằng cách thêm
nhiều HDD hơn, nhưng phải hi sinh chi phí cho điện năng, không gian và chi phí
làm mát.

Xét giá thành cho mỗi GB (Gigabyte), HDD hoàn toàn vượt trội so với SSD. Tuy
nhiên, nếu xét trên toàn hệ thống, ta cần số lượng SSD ít hơn rất nhiều so với
HDD xét trên cùng hiệu năng. Nếu kể thêm các chi phí bảo dưỡng, điện năng
và tiết kiệm không gian, SSD có khả năng là một lựa chọn hấp dẫn hơn
HDD~\cite{wongMarketApplicationsNAND2010}.

Hình~\ref{fig:ssd_vs_hdd} cho thấy tốc độ đọc và ghi vượt trội của SSD so với
HDD. Hơn nữa, trong nhiều năm tới, với công nghệ 3-D NAND, NAND Flash vẫn có khả
năng tiếp tục tăng dung lượng lưu trữ trên cùng một diện tích silicon, làm cho
giá thành liên tục giảm xuống trong tương lai. Trong khi đó, giá thành của HDD
vẫn giữ nguyên trong vòng nhiều năm
nay~\cite{monziocompagnoniReviewingEvolutionNAND2017}.

\image[1]{img/ssd_vs_hdd.png}{So sánh throughput đọc và ghi giữa HDD và
SSD.~\cite{monziocompagnoniReviewingEvolutionNAND2017}}{fig:ssd_vs_hdd}


\subsection{Phase-change memory (PCM)}
Phase-change memory (PCM hoặc PRAM) đã và đang là công nghệ có nhiều hứa hẹn
trong những năm gần đây. Với vai trò là non-volatile memory với độ trễ thấp và
có độ tin cậy cao, PCM được coi là giải pháp có tiềm năng trong các hệ thống bộ
nhớ (memory system). Sự cạnh tranh của PCM với các công nghệ khác dần được tăng
lên khi PCM giảm được kích thước cell, chuyển từ công nghệ bán dẫn sang kích
thước nhỏ hơn là đi-ốt (vertical diode). Chúng giúp giảm giá thành thiết bị và
giảm thiểu chi phí tính toán.

Trong PCM, dữ liệu được lưu bằng cách sử dụng sự chênh lệch điện trở giữa pha
kết tinh có tính dẫn điện (high-conductive crystalline phase) và pha kém kết
tinh không có tính dẫn điện (low-conductive amorphous phase) của các chất liệu
chuyển pha (phase-change material). Dữ liệu được lưu trữ sẽ được đọc bằng cách
đo điện trở của thiết bị PCM. Điểm nổi bật của PCM là dữ liệu có thể được giữ
lại lâu (thông thường là 10 năm ở nhiệt độ phòng) và chỉ mất vài nano giây để
ghi~\cite{galloOverviewPhasechangeMemory2020}.

Mục tiêu của thế hệ PCM đầu tiên là cạnh tranh với DRAM: độ trễ đọc và ghi, độ
bền qua các chu kỳ ghi (cyclic write), retention, v.v. Một vài báo cáo đã cho
ra kết quả khả quan với độ trễ ghi 120ns, độ trễ đọc 150ns và độ bền có chu kỳ
lớn hơn 1E9. Tuy nhiên, PCM vẫn chưa đạt được thành công trên thị trường bởi vì
NAND Flash và DRAM vẫn tiếp tục phát triển nhanh chóng.

Tuy nhiên, với nhu cầu của memory-centric computing và storage-class memory
(SCM), PCM có thể được dùng để lấp khoảng trống giữa NAND Flash và DRAM. Điều
này có nghĩa là SCM phải đồng thời rẻ hơn DRAM và nhanh hơn NAND Flash. Để trở
thành SCM, tốc độ đọc dữ liệu và giá thành của PCM phải tiếp tục được cải thiện
nhiều hơn nữa~\cite{kimEvolutionPhaseChangeMemory2020}.

Công nghệ Cross-point (X-point) được sinh ra để giải quyết các vấn đề trên.
X-point có mật độ cao ($4F^2$) và khả năng chồng lớp 3-D giúp giảm giá thành và
có độ trễ ghi thấp. Intel Optane là bộ nhớ nằm vào phân khúc giữa volatile DRAM
và các thiết bị lưu trữ dữ liệu (chẳng hạn như NAND Flash). Vì Optane có kiến
trúc tổ chức dưới dạng byte (byte-addressable) nên Optane có thể vừa đóng vai
trò là RAM vừa có thể được dùng như các thiết bị lưu trữ dữ liệu như SSD. Intel
Optane DC Persistent Memory Module (hay Optane DC PMM) là bộ nhớ NVDIMM thương
mại đầu tiên. So với các thiết bị lưu trữ PCIe, Optane DC PMM có hiệu năng cao
hơn và hỗ trợ byte-addressable interface. So với DRAM, Optane có mật độ và độ
bền cao hơn, sử dụng ít điện năng hơn khi idle vì không cần phải làm tươi
(refresh for data retention) như
DRAM~\cite{izraelevitzBasicPerformanceMeasurements2019}.

Ngoài các mục đích lưu trữ thông thường, PCM còn có thể hoạt động ở các kiến
trúc non-von Neumann. Ở mô hình điện toán này, các bộ nhớ không chỉ có thể lưu
dữ liệu mà còn có thể thực hiện các tác vụ tính toán. Nhờ vậy, ta không cần
phải chuyển dữ liệu qua lại giữa CPU và RAM, tránh được nút thắt cổ chai trong
hiệu năng của kiến trúc von Neumann~\cite{galloOverviewPhasechangeMemory2020}.

\subsection{Spin-transfer torque magnetic random access memory (STT-MRAM)}
Spin-transfer torque magnetic random access memory (STT-MRAM) là công nghệ bộ
nhớ từ được phát triển dựa trên MRAM (magnetoresistive random access memory,
hay ``bộ nhớ RAM từ điện trở'') với mục tiêu làm tăng khả năng mở rộng
(scalability) của bộ nhớ non-volatile ở các máy điện toán tiên tiến. Các công
nghệ bộ nhớ bán dẫn (semiconductor memory) hiện nay thường sử dụng điện tích
electron (electron charge) mà không chú ý tới một đặc tính khác của electron:
spin cơ học lượng tử (quantum-mechanical spin). Để trở thành một lựa chọn mới
cho bộ nhớ bán dẫn, STT-MRAM phải có mật độ cao ($\sim10F^2$), tốc độ (dưới
10ns cho tác vụ đọc và ghi) và sử dụng ít điện
năng~\cite{apalkovSpintransferTorqueMagnetic2013}.

Ta có thể sử dụng STT-MRAM để lấp đầy các khoảng trống hiệu năng trong kiến
trúc von Neumann. Hình~\ref{fig:mram_architect} đề nghị một kiến trúc bộ nhớ
mới với LL (last level) cache sẽ là STT-MRAM, và hai khoảng trống cũng sẽ được
lấp bởi công nghệ STT-MRAM. Kiến trúc này có thể giải quyết vấn đề về năng
lượng bởi vì ta có thể ngắt điện các bộ nhớ cache non-volatile khi chúng đang
idle. Việc STT-MRAM làm cache cho DRAM cũng là một phương pháp hiệu quả để xóa
bỏ các nút thắt cổ chai hiệu
năng~\cite{hanyuSpintransfertorqueMagnetoresistiveRandomaccess2019}.

\image[1]{img/mram_architect.png}{Trái: kiến trúc bộ nhớ thông thường. Phải:
Kiến trúc bộ nhớ non-volatile sử dụng STT-MRAM (Nguồn: Semiconductor
Engineering).}{fig:mram_architect}