\section{Xu hướng công nghệ non-volatile memory}
\label{chap:experiments}

Kosuke Suzuki~\cite{suzukiSurveyTrendsNonVolatile2015a} đã thực hiện khảo sát
các công nghệ non-volatile từ năm 2000 - 2015. Kết quả đạt được là dữ liệu cho
thấy một bức tranh rõ ràng vẽ lên quá trình phát triển của NAND Flash, PCM,
STT-MRAM và ReRAM. Bài báo so sánh các bộ nhớ non-volatile dựa trên hai tiêu
chí:

\begin{itemize}
    \item \textbf{Thông số kỹ thuật}: So sánh các thông số kỹ thuật của các bộ
    nhớ, chẳng hạn như kích thước cell, tốc độ đọc/ghi của cell, v.v.

    \item \textbf{Thông số kiến trúc}: So sánh hiệu năng của các bộ nhớ đạt
    được ở trong các hệ thống, bao gồm băng thông đọc/ghi và độ trễ truy cập
    tổng quát (overall access latency).
\end{itemize}

\textsc{Cell size} \hspace{0.5cm} Cell size (kích thước cell) là kích thước vật
lý của một cell. Tuy nhiên với cùng cell size, mật độ bit có thể cao hơn nhờ
công nghệ multi-level cell (một cell có thể chứa nhiều hơn một bit) và 3-D
stacking (chồng lớp 3-D). Hình~\ref{fig:survey_cell_size} thể hiện cell size
của từng bộ nhớ non-volatile. PRAM đã giảm xuống còn $4F^2$. ReRAM sử dụng
nhiều kỹ thuật khác nhau để thay đổi điện trở của cell (cell resistance) nên
không thể hiện xu hướng rõ rệt. Tương tự, STT-MRAM cũng có các thiết kế cell
khác nhau. Kích thước cell của NAND vẫn nằm ở quanh $4F^2$ kể từ năm 2000.

\image[1]{img/survey_cell_size.png}{Cell size.}{fig:survey_cell_size}

\textsc{Average cell size} \hspace{0.5cm} Average cell size (kích thước cell
trung bình) tương tự như cell size, nhưng bao gồm cả kích thước mạch ngoại vi
(peripheral circuit) (hình~\ref{fig:survey_average_cell_size}).

\image[1]{img/survey_average_cell_size.png}{Average cell
size.}{fig:survey_average_cell_size}

\textsc{Average bit density} \hspace{0.5cm} Average bit density (mật độ bit
trung bình) được tính bằng dung lượng (Gbit) chia cho kích thước die ($cm^2$).
Hình~\ref{fig:survey_average_bit_density} thể hiện mật độ bit của từng bộ nhớ.
Mật độ bit của PRAM tăng gấp đôi mỗi 1,66 năm và đã đạt 13,5 Gbit/cm2 (2015).
NAND Flash tăng gấp đôi mỗi 1,91 năm và cao nhất là 185,8 Gbit/cm2 (2015).
ReRAM có mật độ bit cao hơn PRAM, và STT-MRAM có mật độ cao nhất là 0.36
Gbit/cm2 (2015).

\image[1]{img/survey_average_bit_density.png}{Average bit
density.}{fig:survey_average_bit_density}

\textsc{Read/write time} \hspace{0.5cm} Thời gian đọc và ghi
(hình~\ref{fig:survey_read_time} và~\ref{fig:survey_write_time}) là thời gian
cần thiết để đọc từ một cell nhưng không bao gồm độ trễ của mạch ngoại vi. Thời
gian đọc tăng đồng thời theo mật độ bit ở tất cả các bộ nhớ trừ STT-MRAM.

\image[1]{img/survey_read_time.png}{Read time.}{fig:survey_read_time}
\image[1]{img/survey_write_time.png}{Write time.}{fig:survey_write_time}

\textsc{Sequential read/write bandwidth} \hspace{0.5cm} Băng thông đọc/ghi tuần
tự được tính với giả thiết kiến trúc DRAM dựa trên DIMM (DRAM-like DIMM-base
architecture) (hình~\ref{fig:survey_sequential_read_bw}
và~\ref{fig:survey_sequential_write_bw}).

\image[1]{img/survey_sequential_read_bw.png}{Sequential read
bandwidth.}{fig:survey_sequential_read_bw}
\image[1]{img/survey_sequential_write_bw.png}{Sequential write
bandwidth.}{fig:survey_sequential_write_bw}