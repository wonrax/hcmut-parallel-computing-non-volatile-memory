\clearpage
\section{Non-volatile memory trong tính toán hiệu năng cao}

% Existing disk-based systems
% can no longer offer timely response due to the high access
% latency to hard disks.

% Moreover, in a
% distributed environment, it is possible to aggregate the
% memories from a large number of server nodes to the extent
% that the aggregated memory is able to keep all the data for a
% variety of large-scale applications (e.g., Facebook [2]). (Zhang)

% It has been shown that simply replacing disk with
% NVRAM is not optimal, due to the high overhead from the
% cumbersome file system interface (e.g., file system cache
% and costly system calls), block-granularity access and high
% economic cost, etc. (Zhang)

Trong thập kỷ qua, số lượng CPU core trên mỗi node không ngừng tăng, có thể từ
vài core cho đến hàng trăm core. Trong khi đó, lượng bộ nhớ RAM cho mỗi core hầu
như không thay đổi. Thông thường, để tăng dung lượng bộ nhớ RAM, ta sẽ phải đầu
tư thêm DRAM với giá thành đắt đỏ. Điều này khiến DRAM chiếm tới 80-90\% chi phí
đầu tư của toàn hệ thống có DRAM pool
lớn~\cite{mironovPerformanceEvaluationIntel2019}.

Đặc điểm của các ứng dụng HPC (high performance computing) là có lượng dữ liệu
khổng lồ và phải thực hiện nhiều tính toán trên tập dữ liệu đó. Với sự phát
triển của công nghệ non-volatile memory như phase change memory hay STT-MRAM,
hiệu năng của các thiết bị này vượt trội so với ổ cứng và có thể tiến lên cạnh
tranh với DRAM. Tính bất khả biến và hiệu năng cao của NVM làm các thiết bị lưu
trữ và bộ nhớ RAM gần nhau hơn bao giờ hết. NVM có hai mô hình sử
dụng~\cite{liuPerformanceEvaluationModeling2017}:

\begin{itemize}
    \item \textbf{Memory-based}: NVM sẽ được sử dụng như DRAM do khả năng có thể
    tổ chức dưới dạng byte (byte-addressable). NVM lúc này được gắn vào bus RAM
    dưới dạng DIMM và được quản lý bởi memory controller. Tuy nhiên, mô hình này
    có nhược điểm là gây phức tạp cho lập trình viên, đặc biệt là khi xử lý tính
    toàn vẹn của dữ liệu khi có sự cố về điện hoặc các trường hợp hệ thống bị
    sập (system failure). Hơn nữa, để khai thác mô hình này, các chương trình
    phải được thiết kế lại để cho phù hợp với kiến trúc bộ nhớ.

    \item \textbf{Storage-based}: NVM có thể sử dụng để thay thế trực tiếp cho
    HDD hoặc SSD và được quản lý bởi I/O controller. Bị giới hạn bởi băng thông
    bus I/O nên mô hình này không thể khai thác hết khả năng của NVM như
    byte-addressability. Tuy nhiên mô hình storage-based có khả năng tương thích
    tối ưu với các hệ điều hành và ứng dụng hiện tại. Người dùng có thể sử dụng
    NVM như một SSD thông thường: phân vùng và cấu hình file system là đã có thể
    trải nghiệm được tốc độ cao của NVM.

\end{itemize}

\subsection{Mô hình storage-based}



Bahn et al.~\cite{bahnImplicationsNVMBased2020} đã tiến hành thử nghiệm ảnh
hưởng của NVM với vai trò là thiết bị lưu trữ dữ liệu (storage device) sử dụng
PCM và STT-MRAM. Hình~\ref{fig:storage_read} và~\ref{fig:storage_write} so sánh
hiệu năng của HDD, RAMDisk (mô phỏng STT-MRAM) và PCMdisk (mô phỏng PCM) khi
chạy IOzone (file system benchmark utility) với nhiều kích thước bộ nhớ RAM khác
nhau. Khi bộ nhớ RAM lớn hơn 1GB, HDD đạt hiệu năng rất cao tại vì tất cả dữ
liệu đã có thể nằm gọn trong buffer cache. Tuy nhiên, với bộ nhớ nhỏ hơn 1GB,
hiệu năng của NVM lớn hơn HDD khá nhiều.

\doubleimage[1]{img/storage_seq_read.png}{Sequential read}{img/storage_rand_read.png}{Random read}{So sánh tốc độ đọc tuần tự và ngẫu nhiên.}{fig:storage_read}

\doubleimage[1]{img/storage_seq_write.png}{Sequential write}{img/storage_rand_write.png}{Random write}{So sánh tốc độ ghi tuần tự và ngẫu nhiên.}{fig:storage_write}

Điều này mang ý nghĩa to lớn đối với việc sử dụng NVM trong tương lai. Kích
thước bộ nhớ RAM hiện tại khó có thể được mở rộng trong khi lượng dữ liệu big
data không ngừng phát triển. Hệ thống tính toán trong tương lai có thể bao gồm
bộ nhớ RAM có dung lượng nhỏ nhưng được hỗ trợ bởi thiết bị lưu trữ NVM để đảm
bảo hiệu năng. Kiến trúc ``small memory, fast storage'' có thể rất hiệu quả
trong các chương trình tiêu tốn bộ nhớ như big data hoặc các nghiên cứu khoa
học~\cite{bahnImplicationsNVMBased2020}.

Liu et al.~\cite{liuPerformanceEvaluationModeling2017} đã thử nghiệm POSIX và
MPI I/O với HDD, SSD và PMBD (Persistent Memory Block Driver, là driver mô phỏng
NVM sử dụng DRAM). Kết quả cho thấy, NVM ít bị ảnh hưởng bởi kích thước page
cache hơn HDD và SSD. Do đó, ta có thể cắt giảm tiêu thụ bộ nhớ của page cache
để giảm bớt chi phí.

Với sự ra đời của các thiết bị Intel Optane, các nhà nghiên cứu đã có thể thực
hiện so sánh hiệu năng của NVM sử dụng thiết bị thật mà không cần phải mô phỏng
(emulate). Thực nghiệm của Izraelevitz et
al.~\cite{izraelevitzBasicPerformanceMeasurements2019} cho thấy khi sử dụng
Optane DC PMM với hệ thống NVMM-aware, hiệu năng có thể được tăng lên đáng kể.

\subsection{Mô hình memory-based}
Các hệ thống lưu trữ hiện tại đã không còn đáp ứng được các nhu cầu về thời gian
do độ trễ quá lớn của ổ cứng. Các công ty lớn như Google, Facebook và Twitter là
một trong những người vấp phải vấn đề này đầu tiên. Tuy nhiên với tình hình hiện
tại, kể cả các công ty vừa và nhỏ có thể gặp trở ngại với hiệu năng của hệ thống
lưu trữ truyền thống. Để đạt được tốc độ "real-time" trong việc truy vấn và xử
lý dữ liệu, ta phải cần đến một hệ thống hay một databse \textit{in-memory}, tức
là có thể giữ được toàn bộ dữ liệu trong RAM.

Với sự phát triển của NVM, NVRAM có thể được lắp đặt cùng vai trò với DRAM ở
memory bus. Lúc này, địa chỉ bộ nhớ (memory address) có thể được chia thành địa
chỉ volatile và địa chỉ non-volatile.

Shanbhag et al.~\cite{shanbhagLargescaleInmemoryAnalytics2020} đã
thực hiện đo lường hiệu năng của mô hình hỗn hợp (hybrid) Intel Optane DC PMM
(App Direct Mode) và DRAM với Star Schema Benchmark (benchmark hiệu năng của
database trong ứng dụng data warehouse). Kết quả cho thấy, mô hình sử dụng NVM
kết hợp DRAM có tốc độ chỉ chậm hơn 1,6 lần so với hệ thống chỉ sử dụng DRAM,
tuy nhiên cho phép tăng dung lượng lưu trữ hơn gấp 10 lần.

Thực nghiệm của Izraelevitz et
al.~\cite{izraelevitzBasicPerformanceMeasurements2019} nhận định rằng, khi sử
dụng Optane DC PMM với cached mode trong các ứng dụng thực tế, NVM có thể đạt
được hiệu năng hoàn toàn so sánh được với DRAM. Mặt khác, với NVM ta có thể tăng
dung lượng bộ nhớ RAM lên nhiều lần.
