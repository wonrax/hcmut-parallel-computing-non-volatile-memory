\section{Non-volatile memory}
\label{chap:non-volatile-memory}

Bộ nhớ (memory) là thiết bị dùng để lưu trữ dữ liệu cho máy tính. Bộ nhớ chính
(primary memory) được dùng cho các tác vụ yêu cầu tốc độ cao, bộ nhớ thứ cấp
(secondary memory) tuy có tốc độ chậm hơn nhưng có thể chứa được nhiều dữ liệu
hơn. Nếu cần thiết, các dữ liệu ở bộ nhớ chính có thể được lưu trữ ở bộ nhớ thứ
cấp thông qua kỹ thuật quản lý bộ nhớ gọi là bộ nhớ ảo (virtual
memory)~\cite{VolatileNonVolatileComputer}.

Trong các hệ thống tính toán ngày nay, bộ nhớ được chia thành hai loại:

\begin{itemize}
    \item \textbf{Volatile memory (bộ nhớ khả biến)}: là loại bộ nhớ cần điện
    năng để duy trì trạng thái các dữ liệu trong bộ nhớ. Đa số các bộ nhớ khả
    biến từ chất bán dẫn rơi vào hai loại: Static RAM (SRAM) hoặc dynamic RAM
    (DRAM). Đối với SRAM, ta chỉ cần duy trì nguồn điện thì dữ liệu sẽ được lưu
    giữ trong bộ nhớ, tuy nhiên SRAM cần sử dụng sáu bán dẫn (transistor) cho
    mỗi bit. Mặt khác, DRAM chỉ cần một bán dẫn cho mỗi bit, qua đó cho phép bộ
    nhớ có mật độ bit cao hơn. Tuy nhiên, DRAM cần các chu trình làm tươi
    (refresh cycle) thường xuyên để lưu giữ dữ liệu.
    
    \item \textbf{Non-volatile memory (bộ nhớ bất khả biến)}: là loại bộ nhớ có
    thể lưu trữ dữ liệu ngay cả khi thiết bị không được cung cấp điện năng.
    Điển hình cho bộ nhớ bất khả biến là các bộ nhớ từ (magnetic storage
    device, ví dụ như ổ cứng hay đĩa mềm), đĩa quang (optical disc) hay NAND
    Flash (điển hình là solid state drive hay SSD). Các bộ nhớ bất khả biến có
    tiềm năng trong tương lai bao gồm FeRAM, CBRAM,PRAM, SONOS, RRAM, Racetrack
    memory, NRAM và Millipede.
\end{itemize}

So với non-volatile memory, volatile memory là công nghệ có tốc độ vượt trội
hơn, tuy nhiên có khả năng mở rộng (scalability) kém hơn. Mục đích của
non-volatile memory là lưu trữ các dữ liệu mà CPU cần truy cập tức thời. Trong
khi đó, non-volatile memory được dùng để lưu trữ các dữ liệu sẽ được sử dụng
trong tương lai.

